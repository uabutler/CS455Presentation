\documentclass{beamer}

\usetheme{Berlin}
%\usetheme{Ilmenau}
\usefonttheme{structurebold}

\title{The Mirai Botnet}
\author{Ulysses Butler, Thu Vo, and Tung Thai, Torey Clark}

\institute{ Truman State University \\ Binary Beasts }

\date{}

%https://www.usenix.org/system/files/conference/usenixsecurity17/sec17-antonakakis.pdf

%You are expected to read and analyze and dissect the information presented in the research paper, and if you look up online you might also be able to find PPT/Video presentation of the paper. Based on your understanding your team would prepare a research presentation that'd include the problem statement (issue), methodology (process), results including pros-cons of the research paper. You are expected to go above and beyond the assigned paper and present your "RESEARCHED" information in a concise manner about the topic.  Any sort of demo is highly encouraged but not required.

%The presentation would last 35 minutes so prepare accordingly and each team member is required to participate in presenting the material. You are also required to prepare a summary document in 800-1200 words, that presents the crux of the research problem and possible solutions. 

%Each team shall have 35 mins to present their research on their topic. You should include the problem statement, motivation, related works, background, contribution, evaluation, significance of the research paper along with your own research on the topic.

\begin{document}

\maketitle

\section{Introduction}

\begin{frame}
	\frametitle{The Paper}
	\begin{itemize}
		\item Title: \textit{Understanding the Mirai Botnet}
		\item \textit{The Proceedings of the 26th USENIX Security Symposium}
		\item This paper explores the Mirai botnet
		\item This botnet was responsible for one of the largest DDoS attacks every recorded.
		\item The purpose of this paper was to learn about how the botnet worked.
		\item Researchers from a number of institutions reversed engineered it to better understand how it spread
		\item This paper then proposes reforms that can be made to prevent this kind of attack in the future
	\end{itemize}
\end{frame}

\begin{frame}
	\frametitle{Contributions}
	\begin{itemize}
		\item<1-> Lead Author
		\begin{itemize}
			\item<1-> Zane Ma - University of Illinois Urbana-Champaign
		\end{itemize}
		\item<2-> This paper had help from many different authors
		\begin{itemize}
			\item<3-> Manos Antonakakis - Georgia Institute of Technology
			\item<3-> Tim April - Akamai Technologies
			\item<3-> Michael Bailey - University of Illinois Urbana-Champaign
			\item<3-> Matthew Bernhard - University of Michigan
			\item<3-> Elie Bursztein - Google
			\item<3-> Jaime Cochran - Cloudflare
			\item<3-> Zakir Durumeric - University of Michigan
			\item<3-> J. Alex Halderman - University of Michigan
		\end{itemize}
	\end{itemize}
\end{frame}

\begin{frame}
	\frametitle{Contributions Cont.}
	\begin{itemize}
		\item<1-> Continued...
		\begin{itemize}
			\item<1-> Luca Invernizzi - Google
			\item<1-> Michalis Kallitsis - Merit Network
			\item<1-> Deepak Kumar - University of Illinois Urbana-Champaign
			\item<1-> Chaz Lever - Georgia Institute of Technology
			\item<1-> Joshua Mason - University of Illinois Urbana-Champaign
			\item<1-> Damian Menscher - Google
			\item<1-> Chad Seaman - Akamai Technologies
			\item<1-> Nick Sullivan - Cloudflare
			\item<1-> Kurt Thomas - Google
			\item<1-> Yi Zhou - University of Illinois Urbana-Champaign
		\end{itemize}
	\end{itemize}
\end{frame}

\section{Spreading}

\begin{frame}
	\frametitle{Bootstraping}
	\begin{itemize}
		\item August 1, 2016: Servers owned by DataWagon began a preliminary scan.
		\begin{itemize}
			\item DataWagon is a bulletproof web hosting provider.
			\item Users are allowed to upload and distribute almost anything using their service.
		\end{itemize}
		\item After this scan, the botnet started infecting computers
		\begin{itemize}
			\item 1 minute -  800 infected devices
			\item 10 minutes - 11,000 infected devices
			\item 20 hours - 65,000 infected devices
			\item Held steady at around 100,000 to 200,000 infections
			\item In December 2016, it peaked at 600,000 devices before beginning to fade
		\end{itemize}
	\end{itemize}
\end{frame}

\begin{frame}
	\frametitle{Spreading}
	\begin{columns}
		\column{0.5\linewidth}
			\includegraphics[width=\textwidth]{fig1.png}
		\column{0.5\linewidth}
			\begin{itemize}
				\item A member of the botnet begins scanning scanning ports on all IPv4 addresses
				\item It scans to find open ports for SSH, Telnet, FTP, and other protocols
				\item It would then use a dictionary attack to brute force into the machine
				\item These were small dictionaries, containing 60 to about 200 credentials
			\end{itemize}
	\end{columns}
\end{frame}

\begin{frame}
	\frametitle{Spreading}
	\begin{columns}
		\column{0.5\linewidth}
			\includegraphics[width=\textwidth]{fig2.png}
		\column{0.5\linewidth}
			\begin{itemize}
				\item The address and credentials of the victim machines where then sent to a report server
				\item This information could later be used by the Command and Control (C2) server
			\end{itemize}
	\end{columns}
\end{frame}

\begin{frame}
	\frametitle{Spreading}
	\begin{columns}
		\column{0.5\linewidth}
			\includegraphics[width=\textwidth]{fig3.png}
		\column{0.5\linewidth}
			\begin{itemize}
				\item The records server then sent this information to the load program
				\item This program would download a binary onto the victim and run the program
			\end{itemize}
	\end{columns}
\end{frame}

\begin{frame}
	\frametitle{Loading the Binary}
	\begin{itemize}
		\item The aforementioned loader program downloaded an architecture specific binary
		\item The machine would then run the binary and change the process information to make it harder to detect
		\item The binary is then deleted.
		\begin{itemize}
			\item This means infections won't carry across reboots
		\end{itemize}
		\item Once the victim is infected, it starts scanning
		\begin{itemize}
			\item It would specifically avoid scanning servers owned by major corporations or the government
			\item These entities would likely be too secure for this simple attack
			\item This also allowed the bot to keep a lower profile
			\item These organizations would be much more likely to start search for and exploiting weaknesses in the malware if it infected their machines
		\end{itemize}
	\end{itemize}
\end{frame}

\section{Exploiting IoT}

\begin{frame}[fragile]
	\frametitle{Internet of Things Security}
	\begin{itemize}
		\item The Internet of Things
		\begin{itemize}
			\item Includes security cameras, routers, network-access storage, TV receivers, printer, DVRs, etc.
			\item Typically embedded systems that aren't powerful
		\end{itemize}
		\item Manufactures neglect security
		\begin{itemize}
			\item Many manufactures use one user name and password
			\item Common passwords are frequent. \verb|password|, \verb|admin|, etc.
			\item Some devices even have credentials hard coded in firmware
			\item Most companies don't have the infrastructure to release patches for these systems
		\end{itemize}
		\item This allowed the bot to easily infect a large number of machines
	\end{itemize}
\end{frame}

\begin{frame}
	\frametitle{Disadvantages}
	\begin{itemize}
		\item These less powerful devices also hurt Mirai's growth
		\begin{itemize}
			\item Mirai had a doubling time of 75-minutes
			\begin{itemize}
				\item Compare to 37-minutes for Code-Red
				\item 9-minutes for Blaster
			\end{itemize}
			\item Most bots scanned at less than 250 bytes per second
			\item Much slower than other bot nets
			\item SQL Slammer was about 6000 times faster at 1.5 megabytes per second
		\end{itemize}
		\item Most devices were found in low bandwidth countries
		\begin{itemize}
			\item Most infected devices were from South America and South-east Asia
			\item Brazil, Colombia, and Vietnam hosted most of the bots
		\end{itemize}
	\end{itemize}
\end{frame}

\section{Attacking}

\begin{frame}
    \frametitle{How DDoS Works}
    \begin{itemize}
        \item How to DDoS for Dummies
            \begin{itemize}
                \item A DDoS seeks to restrict a servers capabilities to respond to users by flooding it with requests from multiple different machines.
                \item DDoS attacks are more difficult to protect against compared to a DoS attack
                \item It is difficult to blacklist multiple IP Addresses
                \item It's nearly impossible to distinguish between real requests and the attack.
            \end{itemize}
    \end{itemize}
\end{frame}

\begin{frame}
    \frametitle{Volumetric Attacks}
        \begin{itemize}
            \item Volumetric Attacks consist of a flooding a server with request packets
            \item Overwhelms its ability to respond
            \item Requires work to generate a high count of requests
            \item With requests from multiple machines, it is difficult to prevent or dampen an attack on a server.
    \end{itemize}
\end{frame}
\begin{frame}
    \frametitle{Protocal Attacks}
     \begin{itemize}
            \item Protocol Attacks seek to disable a server by exploiting a weakness in a given protocol
            \item SYN floods attacks TCP by exploiting the three-way handshake process to create a backlogged queue
            \item Ping attacks uses a large number of pings to attack a server
            \item UDP floods send massive amounts of packets to random ports to overwhelm the queue of responses
        \end{itemize}
\end{frame}

\begin{frame}
    \frametitle{Application Layer Attacks}
        \begin{itemize}
            \item Application layer attacks attempt to exploit the layer of human interaction with a machine
            \item Nearly indistinguishable from real user interaction
            \item requires far less resources to execute this attack than it takes to prevent that attack
            \item This makes these attacks resource efficient for an attacker
    \end{itemize}
\end{frame}

\begin{frame}
    \frametitle{Types of Services Attacked}
    \begin{itemize}
        \item Most attacks were orchestrated against targets in the United States(50.3\%), France(6.6\%), and the UK(6.1\%).
        \item Mirai could also target particular ports to affect specific services
        \begin{itemize}
        		\item The most common ones attacked were 80(HTTP, 37.5\%)
        		\item 25565(Minecraft, 9.2\%)
        		\item 443(HTTPS, 6.4\%)
        		\item and 23594(Runescape, 3.4\%)
        \end{itemize}
        \item Several Mirai C2 servers were attacked by some of its other C2 servers
        \item These were likely from renting DDoS attackers against other renting DDoS attackers.
    \end{itemize}
\end{frame}

\begin{frame}
    \frametitle{Attacks}
	\begin{itemize}
		\item General Targets by Mirai
			\begin{itemize}
				\item Multiple DDoS attacks against a variety of targets
				\item Game Servers (primarily Minecraft and Runescape)
				\item Political WebsitesA
				\item Anti-DDoS services
			\end{itemize}
		\item Notable Targets
			\begin{itemize}
				\item Krebs on Security
				\begin{itemize}
					\item This was a high-profile attack on a well-known security blog
					\item The attack peaked at around 600 Gbps, the largest Akamai had ever seen 						\item This forced them to drop Krebs as a client due to high costs
				\end{itemize}
				\item Dyn - DNS attack disrupted access for Amazon, Github, Netflix, Twitter, and others
				\item Lonestar Cell - most attacked target, destroyed internet capabilities in Liberia
			\end{itemize}
	\end{itemize}
\end{frame}

\section{Methodology}

\begin{frame}
	\frametitle{Methodology}
	\begin{itemize}
		\item The researches and authors of this paper used a number of techniques
		\item They used attempted to monitor the botnet's spread
		\item Many binaries used by the malware were captured
		\item A number of organizations tried a variety of techniques and shared their information for this paper.
	\end{itemize}
\end{frame}

\begin{frame}
	\frametitle{Network Telescope}
	\begin{itemize}
		\item One method for monitoring the spread was by using network telescopes
		\item Purpose: to analyze the growth and size of the botnet
		\item Monitored network request (scan) to a network telescope composed 4.7 millions IP address
		\begin{itemize}
			\item On average, the network telescope received 1.1 million packets from 269,000 IP addresses per minute 
			\item Observed 116.2 billions Mirai probes from 55.4 millions IP address
		\end{itemize}
		\item A raw count of IP addess is a poor metric due to DHCP churn
		\begin{itemize}
			\item Consider the number of hosts actively "scanning" at the start of every hours
			\item Identified scans that targeted the IPv4 address space at an estimated rate of at least five packets per second
		\end{itemize}
	\end{itemize}
\end{frame}

\begin{frame}
	\frametitle{Active Scanning}
	\begin{itemize}
		\item The researches also tried scanning infected devices
		\item Purpose: to analyze infected device composition (manufacturer and model).
		\item Focus on scans of HTTPS, FTP, SSH, Telnet, and CWMP.
		\item Difficulties and challenges to make accurate device labeling:		
		\begin{itemize}
			\item Mirai prevents infected devices from being scanned
			\item The scan often takes 24 hours to complete, during which devices may churn to a new IP address
			\item Resolution: restricting analysis to banners that were collected within twenty minutes of scanning activity
		\end{itemize}
		\item Post-filtering, the dataset include 1.8 millions banner associated with 1.2 million IP address
		\begin{itemize}
			\item Process each banner to identify the device manufacturer and model using Nmap
			\item In total, identified 31.5 \% of banners (about 600k banners)
		\end{itemize}
	\end{itemize}
\end{frame}

\begin{frame}
	\frametitle{Telnet Honeypots}
	\begin{itemize}
		\item Use a set of Telnet honeypots that masqueraded as vulnerable IoT
		\begin{itemize}
			\item The honeypot logged all incoming traffic and downloaded any binaries that the attackers attemps to install
			\item Block all outgoing request to avoid collateral damages
		\end{itemize}
		\item Logged 80K connection attempts from 54K IP addresses and collected 141 unique binaries.
		\begin{itemize}
			\item Supplemented these data with unique binaries from others
			\item In totals, they collected 1028 unique binaries
		\end{itemize}
		\item Analyzed the most common, binaries for MIPS 32-bit, ARM 32-bit, and x86 32-bit.
		\begin{itemize}
			\item Extracted the set of logins, password, IP blacklists, and C2 domains
			\item Identified 67 C2 domains and 48 distinct username password dictionaries (containing a total 371 unique passwords)
		\end{itemize}
	\end{itemize}
\end{frame}

\begin{frame}
	\frametitle{Active \& Passive DNS}
	\begin{itemize}
		\item Purpose: to construct a graph reflecting the shared infrastructure used by Mirai
		\item Collected 209 millions passive DNS record per day (historical record of DNS zone) and 290 millions active DNS record per day of C2 server
		\item Use above dataset to identify shared DNS infrastructure by linking related historic domain names (RHDN) and related historic IPs (RHIPs)
		\begin{itemize}
			\item For a given C2 domain, identify the IP address it previously resolved to and added them to a growing set of domains and IPs
			\item Starting from an IP and finding any domain names that concurrently resolved it
			\item In the end, from a single domain name, we can expand a set of domain name and IP addresses
		\end{itemize}
	\end{itemize}
\end{frame}

\begin{frame}
	\begin{figure}
		\includegraphics[width=50 mm, scale = 0.5]{C2DomainRelationship.png}
		\caption{ \textbf{C2 Domain Relationships} -- We visualize related C2 infrastructure, depicting C2 domains as nodes and shared IPs as edges between two domains.}
	\end{figure}
\end{frame}

\begin{frame}
	\frametitle{Attack Commands}
	\begin{itemize}
		\item Purpose: to track the attack commands issued by the Mirai operators
		\item Simulated a Mirai-infected device and communicated with the C2 server using a custom bot-to-C2 protocol
		\item In total, Akamai observed 64K attack commands issued by 484 unique C2 servers  (by IP address)
		\begin{itemize}
			\item This is a naive analysis because individual C2 servers often repeat the same attack command in rapid succession
			\item Resolution: collapse matching commands that occur within 90 seconds of each others
			\item Results: 15,194 attacks from 146 unique IP clusters, which cover the Dyn attack and Liberia attacks
		\end{itemize}
	\end{itemize}
\end{frame}

\begin{frame}
	\frametitle{DDoS Attack trace}
	\begin{itemize}
		\item Purpose: to corroborate the IP addresses observed in attacks versus those found scanning our network telescope.
		\begin{itemize}
			\item Dyn provided 107.5K IP addresses associated
with attack on October 21, 2016 and 158.8K IP addresses involved in attack on September 25, 2016
			\item Form a statistics to calculate what fraction of these IP addresses matched the list of IP address obseved by our network telescope
		\end{itemize}
	\end{itemize}
\end{frame}

\section{Open Source}

\begin{frame}
	\frametitle{Releasing the Source Code}
	\begin{itemize}
		\item The source code was released on September 30, 2016
		\item A user named ``Anna-senpai'' released the source code for free on hackerforums.net
		\item This spawned a number of copycat attacks with variations on the original bot
		\begin{itemize}
			\item Many included new exploits, modified dictionaries, and different IP blacklists
		\end{itemize}
		\item These botnets eventually starting competing
		\begin{itemize}
			\item Killing processes started by similar bots
			\item Closing the ports used to attack the machine
			\item At various points, competing command and control servers were subject to DDoS attacks
		\end{itemize}
	\end{itemize}
\end{frame}

\section{Defending}
\begin{frame}
    \frametitle{Defense Against the Dark Arts}
    \begin{itemize}
        \item Industry Improvements
            \begin{itemize}
                \item Randomized default passwords 
                \item Closed ports on default
                \item Employ Automatic Updates and Self-Monitoring
                \item Creating Standard for Model and Firmware identification
            \end{itemize}
        \item User Improvements
            \begin{itemize}
                \item Create good secure credentials and not use the defaults
                \item Purchase from reputable and secure companies
                \item Replace old and unsupported devices
            \end{itemize}
    \end{itemize}
\end{frame}

\begin{frame}
    \frametitle{Defense Against the Dark Arts}
    \begin{itemize}
        \item Randomized default passwords prevent attackers from employing a dictionary of default passwords.
        \item Having ports not used default to closed mitigates the chances of a successful attack.
        \item Automatic updates prevent users from refusing updates during hours of use and keeps systems secure against previous exploits. Bug bounties encourage the community to find and report all possible exploits to be patched.
        \item Standards for model and version identification allow server admins to easily see any and all machines that have known vulnerabilities.
    \end{itemize}
\end{frame}
\begin{frame}
    \frametitle{Defense Against the Dark Arts}
    \begin{itemize}
        \item Users should create secure usernames and passwords for all devices to mitigate the chance of it being hacked using brute force.
        \item Smart purchases from known and trusted companies that prioritize security of their manufactured devices acts as a deterrent from would be attackers.
        \item Old and unsupported devices should be replaced with newer models that conform with current security standards and have strong customer support.
    \end{itemize}
    
\end{frame}

\section{Further Research}

\begin{frame}
	\frametitle{Origin of Mirai}
	\begin{itemize}
		\item After this paper was released, we started to learn more about the botnet's origins
		\item Mirai was created by 3 computer science students
		\begin{itemize}
			\item 21-year-old Paras Jha
			\item 20-year-old Josiah White
			\item 21-year-old Dalton Norman
		\end{itemize}
		\item It's likely named after the anime Mirai Nikki (Future Diary)
		\item As revealed after the trial, this botnet was originally created to DDoS Minecraft servers
		\item The group allegedly started trying to profit from their botnet
		\begin{itemize}
			\item Extorting profitable servers
			\item Creating a company to protect server owners from DDoS attacks
			\item Renting out their botnet to other cybercriminals
		\end{itemize}
	\end{itemize}
\end{frame}

\begin{frame}
	\frametitle{Investigations}
	\begin{itemize}
		\item Anna-senpai boosted about his programming skills on the forum
		\begin{itemize}
			\item After investigating companies related to DataWagon, Krebs found ProTraf solutions
			\item The skills Anna-senpai had were extremely similar to Jha's.
			\item He also has experience running Minecraft servers, similar to the original targets
		\end{itemize}
		\item According to Krebs on Security, the creators then released the source code to ``distance themselves from their creation''
		\item Many attacks, such as the attack on Dyn, are believed to be a result of copy cat attackers
	\end{itemize}
\end{frame}

\begin{frame}
	\frametitle{Investigation}
	\begin{itemize}
		\item On December 8, 2017, the FBI announced that it had secured guilty pleas
		\item The authors were able to avoid jail time after cooperating with the FBI to help track other cybercriminals.
		\item This allowed them to avoid jail time
		\begin{itemize}
			\item Five-year probation
			\item 2,500 Hours of community service
			\item \$127,000 in restitution
			\item They had to forfeit the concurrency made during that time
			\item This was about 17 bitcoin
		\end{itemize}
	\end{itemize}
\end{frame}

\section{Conclusion}

\begin{frame}
	\frametitle{Conclusion}
	\begin{itemize}
		\item Mirai was a centralized botnet that was responsible for some of the largest DoS attacks every recorded.
		\item This bot was relatively simple with small dictionaries
		\item They creators of this bot exploited the security negligence of hardware manufactures
		\item They were able to quickly take over a large number of IoT devices
		\item This attack served as a wake up call, prompting reform in these industries
	\end{itemize}
\end{frame}

\end{document}
