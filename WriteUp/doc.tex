\documentclass[conference]{IEEEtran}
%\IEEEoverridecommandlockouts
\usepackage{cite}
\usepackage{amsmath,amssymb,amsfonts}
\usepackage{algorithmic}
\usepackage{graphicx}
\usepackage{textcomp}
\usepackage{xcolor}

\usepackage{lipsum}

\begin{document}

\title{Understanding the Mirai Botnet: A Summary}

\author{\IEEEauthorblockN{Ulysses Butler, Torey Clark, Thu Vo, Tung Thai}
\IEEEauthorblockA{\textit{CS 455 - Cybersecurity Fundamentals} \\
\textit{Truman State University}\\
Kirksville, MO \\
ub4782@truman.edu}
}

\maketitle

\section{Introduction}
In August of 2016, a server from a US bulletproof hosting service starting scanning computers. Shortly after this preliminary scan, the Marai botnet emerged. Within hours, it had infected hundreds of thousands of poorly secured IoT devices. In this paper, Zane Ma et al. reverse engineer the malware and try to understand how it spread and what made it so effective. This botnet gained notoriety after being used to commit record smashing DDoS attacks. This botnet, and others that evolved from it, are believed to be responsible for high profile attacks on Krebs on Security, Dyn, OVH, and Lonestar Cell. Services provided by companies like Amazon, Twitter, GitHub, and Netflix were disrupted since they receive DNS services from Dyn. By the end of 2017, the botnet started to fade away and the creators were located by Krebs and the FBI.

\begin{figure}[b]
\centerline{\includegraphics[width=0.45\textwidth]{../fig1.png}}
\caption{A bot would start scanning other devices to infect}
\label{scan}
\end{figure}

\section{Spreading Marai}
On August 1st, 2016, servers owned by DataWagon began a scan that lasted about two hours. About 40 minutes later, it starting infecting devices. Within a minute, over 800 devices were infected. After 10 minutes, they had over 11,000 devices with the malware. This kept increasing with a doubling rate of about 75 minutes. Once infected, a device would begin to scan other devices. It would check every IPv4 address for open ports running services like SSH, FTP, and Telnet. It carried a blacklist of IP addresses associated with the US Department of Defense and a number of major corporations. This allowed it to go largely undetected. Furthermore, these entites has the resources to properly study and defeat the botnet, so avoiding them gave Marai a better chance of spreading. Once it found a device to infect, it would start brute forcing the credentials. It did this using a small dictionary of anywhere from 60 to about 200 default manufacturer credentials as seen in fig.~\ref{scan}. 

 \begin{figure}[t]
\centerline{\includegraphics[width=0.45\textwidth]{../fig2.png}}
\caption{The IP address and credentials are then sent to the report server}
\label{report}
\centerline{\includegraphics[width=0.45\textwidth]{../fig3.png}}
\caption{The report server dispatches the loader program to install the binary}
\label{loader}
\end{figure}

Once the bot figured out what the username and password were, this information was sent back to a report server, as seen in fig.~\ref{report}. This server would then send the information to dispatch the loader program. The loader program would then log in to the newly cracked machine and download an architecture specific binary. This binary would then be run. Once the program was loaded into memory, it would delete the binary. This made it much more difficult to study. It would then obscure infomation about its process to make it harder to detect, shown in fig.~\ref{loader}.

Once the bot figured out what the username and password were, this information was sent back to a report server. This server would then send the information to dispatch the loader program. The loader program would then log in to the newly cracked machine and download an architecture specific binary. This binary would then be run. Once the program was loaded into memory, it would delete the binary. This made it much more difficult to study. It would then obscure infomation about its process to make it harder to detect. 

Once the bot figured out what the username and password were, this information was sent back to a report server. This server would then send the information to dispatch the loader program. The loader program would then log in to the newly cracked machine and download an architecture specific binary. This binary would then be run. Once the program was loaded into memory, it would delete the binary. This made it much more difficult to study. That being said, this would result in the malware being purged upon a system reboot. It would then obscure infomation about its process to make it harder to detect.  Once this was done, the new bot would start the scanning process again, and await orders from the command and control (C2) server.

\section{Internet of Things Devices}

This botnet was notable for targeting devices in the Internet of Things (IoT). This mainly consists of small, embedded devices. This includes things like cameras, printers, routers, and DVRs. That being said, the Internet of Things doesn't include devices like personal computers, servers, and cell phones. These devices are much more powerful, but also much more secure. Manufactures of IoT devices often neglect good security practice. Many of these device ship with the same username and password. In many cases, they use common credentials like \verb|admin|, \verb|password|, and the name of the company or product. This is how the botnet was able to spread so far despite having such a small credential list.

\bibliographystyle{IEEEtran}
\bibliography{ref}

\end{document}
