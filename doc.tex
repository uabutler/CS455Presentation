\documentclass{beamer}

\usetheme{Berlin}
\usefonttheme{structurebold}

\title{The Mirai Botnet}
\author{Torey Clark, Ulysses Butler, Thu Vo, and Tung Thai}

\institute{ Truman State University \\ Binary Beasts }

\date{}

\begin{document}

\maketitle

\section{Introduction}

\begin{frame}
	\frametitle{The Paper}
	\begin{itemize}
		\item<+-> Title: Understanding the Mirai Botnet
		\item<+-> Journal: The Proceedings of the 26th USENIX Security Symposium
	\end{itemize}
\end{frame}

\section{Spreading Mirai}

\begin{frame}
	\frametitle{Bootstraping}
	\begin{itemize}
		\item<+-> August 1, 2016: Servers owned by DataWagon began a preliminary scan.
		\begin{itemize}
			\item<+-> DataWagon is a bulletproof web hosting provider.
			\item<+-> Users are allowed to upload to upload and distribute almost anything using their service.
		\end{itemize}
		\item<+-> After this scan, the botnet started infecting computers
		\begin{itemize}
			\item<+-> 1 minute -  800 infected devices
			\item<+-> 10 minutes - 11,000 infected devices
			\item<+-> 20 hours - 65,000 infected devices
			\item<+-> Held steady at around 100,000 to 200,000 infections
			\item<+-> In December 2016, it peaked at 600,000 devices before beginning to fade
		\end{itemize}
	\end{itemize}
\end{frame}

\begin{frame}
	\frametitle{Spreading}
	
\end{frame}

\section{Exploiting IoT Devices}

\begin{frame}[fragile]
	\frametitle{Internet of Things Security}
	\begin{itemize}
		\item<+-> The Internet of Things
		\begin{itemize}
			\item<+-> Includes security cameras, routers, network-access storage, TV receivers, etc.
			\item<+-> Typically embedded systems that aren't powerful
		\end{itemize}
		\item<+-> Manufactures neglect security
		\begin{itemize}
			\item<+-> Many manufactures use one user name and password
			\item<+-> Common passwords are frequent. \verb|password|, \verb|admin|, etc.
			\item<+-> Some devices even have credentials hard coded in firmware
			\item<+-> Most companies don't have the infrastructure to release patches for these systems
		\end{itemize}
		\item<+-> This allowed the bot to easily infect a large number of machines
	\end{itemize}
\end{frame}

\begin{frame}
	\frametitle{Disadvantages}
	\begin{itemize}
		\item<+-> These less powerful devices also hurt Mirai's growth
		\begin{itemize}
			\item<+-> Mirai had a doubling time of 75-minutes
			\begin{itemize}
				\item<+-> Compare to 37-minutes for Code-Red
				\item<+-> 9-minutes for Blaster
			\end{itemize}
			\item<+-> Most bots scanned at less than 250 bytes per second
			\item<+-> Much slower than other bot nets
			\item<+-> SQL Slammer was about 6000 times faster at 1.5 megabytes per second
		\end{itemize}
		\item<+-> Most devices were found in low bandwidth countries
		\begin{itemize}
			\item<+-> Most infected devices were from South America and South-east Asia
			\item<+-> Brazil, Colombia, and Vietnam hosted most of the bots
		\end{itemize}
	\end{itemize}
\end{frame}

\section{Attacking}
\begin{frame}{Attacks}
	\begin{itemize}
		\item<+->Distributed Denial of Service (DDoS) Strategies
			\begin{itemize}
				\item<+->Volumetric(32.8\%)
				\item<+->TCP state exhaustion(39.8\%)
				\item<+->Application layer attacks(34.5\%)
			\end{itemize}
		\item<+->Targets
			\begin{itemize}
				\item<+->Krebs on Security - largest reported DDos attack to that point
				\item<+->Dyn - DNS attack disrupted access for Amazon, Github, Netflix, Twitter, and others
				\item<+->Lonestar Cell - most attacked target, destroyed internet capabilities in Liberia
			\end{itemize}
	\end{itemize}
\end{frame}
	\begin{itemize}
		
	\end{itemize}
\begin{frame}
	
\end{frame}

\section{Open Source Software}

\section{Defending Against Mirai}

\end{document}
